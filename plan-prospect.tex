\input{preable}

\begin{document}

\begin{titlepage}
	\centerline{\includegraphics[width=2cm]{dstu-logo}}
	\vfill
	%\Large
	\centerline{МИНЕСТЕРСТВО НАУКИ И ВЫСШЕГО ОБРАЗОВАНИЯ}
	\centerline{РОССИЙСКОЙ ФЕДЕРАЦИИ}
	\vfill
	\centerline{\bf ФЕДЕРАЛЬНОЕ ГОСУДАРСТВЕННОЕ БЮДЖЕТНОЕ}
	\centerline{\bf ОБРАЗОВАТЕЛЬНОЕ УЧРЕЖДЕНИЕ ВЫСШЕГО ОБРАЗОВАНИЯ}
	\centerline{\bf «ДОНСКОЙ ГОСУДАРСТВЕННЫЙ ТЕХНИЧЕСКИЙ УНИВЕРСИТЕТ»}
	\centerline{\bf (ДГТУ)}
	\normalsize
	\vfill\vfill
	\centerline{Факультет «Отдел аспирантуры и докторантуры»}
	\centerline{Кафедра «Кибербезопасность информационных систем»}
	\vfill
	%\rightline{Место для подписи завкафедрой}
	\vfill
	%\Large{\centerline{\bf ПОРТФОЛИО АСПИРАНТА}}
	\centerline{План-проспект реферата по дисциплине «История и философия науки»}
%	\centerline{«История и философия науки»}
	\centerline{на тему «История и философия криптографии»}
	\vfill
	\vfill
	\vfill
	\vfill
	\vfill
	\vfill
	\vfill
	\rightline{Выполнил:}
	\rightline{аспирант 1го курса}
	\rightline{по направлению подготовки}
	\rightline{09.06.01 «Информатика и вычислительная техника»}% номер зачетной книжки: 2142748
	\rightline{Галов К.А.}
	\rightline{Проверил: д.ф.н., доцент Тазаян А.Б.} % Тазаян Араван Бабкенович
	\vfill
	\vfill
	
	\centerline{г. Ростов-на-Дону}
	\centerline{2022 г.}
	
\end{titlepage}
\setcounter{page}{2}

\section*{Введение}
Определение криптографии. Актуальность этого раздела науки. Цель криптографии.

\section{Идеалы и нормы научности в криптографии}
Принципы написания научных работ по криптографии. Идеализация участников канала связи.
% обычные принципы + то, что злоумышленник ограничен только законами физики.

Литература: \cite{phil2016nesm, scarani2004quantum}

\section{Древнейшие шифры}
Шифрование в древнем Риме. Особенности первых методов шифрования. 

Литература: \cite{senthil2013modern}

\section{Средневековые шифры}
Шифры времен средневековья. Шифр Виженера. 

Литература: \cite{omran2011cryptanalytic}

\section{Криптография времен второй мировой войны}
Шифровальная машина Энигма. Работы Алана Тьюринга. Шифровальная техника СССР.

Литература: \cite{d2006sov, g2005n, hodges2014alan}


\section{Проблемы современной криптографии}
Особенности современной криптографии. Теоретическая ненадежность современной криптографии.

Литература: \cite{bellare2005introduction}

\section{Квантовая криптография}
Особенности квантовой криптографии. Её надежность.

Литература: \cite{scarani2004quantum, ch1984quantum, lo2005decoy}

\section*{Заключение}
Общий вывод об истории криптографии, как науки и обзор дальнейших перспектив развития.

\bibliography{refs}

\clearpage
\pagenumbering{gobble}
\section*{Отзыв научного руководителя на тему реферата}
Я, Черкесова Лариса Владимировна, научный руководитель аспиранта первого года обучения Галова Кирилла Алексеевича подтверждаю, что данная тема соответствует специальности 09.06.01 «Информатика и вычислительная техника» и направлению исследований аспиранта.
\bigskip

\rightline{проф. каф. «КБИС», д.ф.-м.н. \hspace{0.5cm} \underline{\includegraphics[width=3cm]{cherlv-podpis}} \hspace{0.5cm} Черкесова Л. В}

\end{document}